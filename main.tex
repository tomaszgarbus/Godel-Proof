\documentclass{article}
\usepackage{mathtools,amssymb}
\usepackage{graphicx} % Required for inserting images
\usepackage[a4paper, total={7in, 10in}]{geometry}

\title{A \textbf{boring} guide through Gödel's theorems}
\author{Tomek Garbus (github.com/tomaszgarbus)}
\date{July 2024}

\begin{document}

\maketitle

\section{TODOs}

\begin{itemize}
    \item Replace ' with $^+$ as the successor symbol
    \item Cleaner division where we define things
    \item Explain the logical formulas for defining properties
\end{itemize}

\section{Introduction}
This article aims to provide an accessible and complete guidance
through the definition and proof of Gödel's incompleteness theorems, as well
as all the prerequisites along the way.

We will not discuss philosophical implications of the theorems nor their
historical context. Not because they are not worth our attention, quite the contrary.
I am assuming that you, dear Reader, already have your motivation for understanding the
technicalities of Gödel's theorems just as I had mine.

\section{Outline}

I promised you a concise guide without unnecessary rambling. This is the only section where
I break this promise to introduce some ideas, so bear with me.

\subsection{Self-reference}

Gödel's proof is built around the notion of \textit{self-reference} of logical statements.
This is all you need to know, but if you bear with me for just a few paragraphs, I will try
to paint a fuller picture of what kind of self-reference we are talking about.

Let's consider a few other examples of self-reference in other domains:

\begin{enumerate}
    \item Well-known paradox statement \textit{This sentence is false}.
    \item Utterance \textit{I like bread.}
    \item A computer program copying its own executable to another block of memory.
    \item A \textit{quine} -- computer program reproducing its own source code.
\end{enumerate}

If you pay close attention to these examples, you will notice that one of them stands out.

The first three can be considered a \textit{pointing} self-reference. The sentence \textit{this
sentence is false} is pointing to itself with the word \textit{this}. When I say \textit{I
like bread}, I'm pointing at myself with the word \textit{I}. A program copying its own executable
probably stores a pointer to the memory block containing its binary.

A quine clearly stands out. We can call it, for a lack of a better term, \textit{full}
self-reference or \textit{value} self-reference. In other words, not only are the referring
agent and the referent identical (same as in pointing self-reference), but they are fully
recoverable from the \textit{reference itself}. The reference itself contains everything there
is to know about the referent.

\subsection{Meaning--representation duality}

Every computer program, every statement in logic, every natural language sentence
can be seen through two different modalities -- \textit{meaning} and \textit{representation}.
This is pretty self-explanatory. For instance, for a C++ program, the representation is a text
file on disk, and its meaning is actualized in the process of compiling and running the
executable.

In the examples of value self-reference from the previous section, meaning and representation
are very closely tied. The \textit{meaning} of a quine is precisely that it produces its own
\textit{representation}. A program that copies its own executable also manipulates its own
representation -- but in this case, its representation is in binary, not text form.

In everyday life we seldom mix between the two. We value programs, logic propositions
and sentences for their meaning, and use their representation to disseminate them.

The most common example of mixing these two modalities I can think of are utterances of
meta-language, for example: \textit{In Polish, the word computer is spelled 'komputer'}.
Even when we read the whole meta-language sentence for its meaning, we focus on the
representation in the object language part of the sentence.

Gödel's proof is really an unusual exercise in hopping between meaning and representation
of logic propositions. It doesn't feel natural at first -- and for me personally it was
the reason why I felt the need to follow Gödel's theorems' proof in detail rather than
taking the general outline at face value.

\subsection{Finally, outline}

The construction of Gödel's theorems will follow through three main steps:

\begin{enumerate}
    \item Create a numerical representation $\ulcorner ... \urcorner$ for logic propositions,
        such that for every logic formula $A$, $\ulcorner A \urcorner$ is a unique natural number:
        $\forall_{A_1, A_2} (\ulcorner A_1 \urcorner \neq \ulcorner A_2 \urcorner
        \rightarrow A_1 \neq A_2$). This is called Gödel numbering.
    \item Introduce a property $Prov_F(n)$ which is true if the formula with Gödel number $n$
        is provable within formal system $F$.
    \item Prove that in a chosen (sufficiently powerful) arithmetic system $F$,
        it is possible to construct \textit{Gödel formula} $G_F$ such that
        $F \models G_F \leftrightarrow \neg Prov_F(\ulcorner G_F \urcorner)$. In other words, within system
        $F$, the Gödel formula $G_F$ says: \textit{$G_F$ is not provable within $F$}.

        Gödel's formula $G_F$ can neither be proved nor disproved within $F$ and in this
        sense it is a formal version of the sentence \textit{This sentence is false.}\footnote{
            A careful reader will point out that falsity and not being provable are two different
            properties. Thus Gödel's formula is not literally equivalent to the
            paradox sentence, but the analogy is still helpful.
        }
\end{enumerate}

\section{Gödel theorems statement}

Just a quick recap of what we're going to prove:

\subsection{First incompleteness theorem}

\textit{Any consistent formal system $F$, within which a certain amount of elementary
arithmetic can be carried out is incomplete; i.e. there are statements of $F$ which can be
neither proved nor disproved in $F$.}

\subsection{Second incompleteness theorem}
\textit{For any consistent system $F$ within which a certain amount of elementary
arithmetic can be carried out, the consistency of $F$ cannot be proved within $F$ itself.}

\section{Gödel numbering}

\subsection{Symbol numbers}

We will start with defining \textit{symbol numbers}, that is mappings from elements of
logic syntax to natural numbers:

\begin{center}
{\renewcommand{\arraystretch}{1.5}
\begin{tabular}{ c c c c c }
 $\#('0') = 1$ & $\#(''') = 2$ & $\#('+') = 3$ & $\#('\cdot') = 4$ & $\#('=') = 5$ \\ 
 $\#('(') = 6$ & $\#(')') = 7$ & $\#('\rightarrow') = 8$ & $\#('\neg') = 9$ & $\#('\forall') = 10$ \\ 
 & & $\#('x_i') = 11 + i$ & & \\ 
\end{tabular}
}
\end{center}

Note that we are using a minimal arithmetic where natural numbers are encoded as successors of
(successors of) zero:
\begin{center}
    $1 := 0'$\\
    $2 := 1' = 0''$\\
    ...\\
    $n := 0{\overbrace{'''...'}^{n}}$
\end{center}

\subsection{Coding sequences}

A sequence $(n_0, n_1, ..., n_k)$ can be coded as $p_0^{n_0} \cdot p_1^{n_1} \cdot ... \cdot p_n^{n_n}$,
where $p_i$ are next prime numbers and for $n_i$ we use symbol numbers.

For example, expression $'0=0'$ will be coded as
$p_0^{\#('0')} \cdot p_1^{\#('=')} \cdot p_2^{\#('0')} = 2^1 \cdot 3^5 \cdot 5^1 = 2430$.

Gödel number for formula $A$ is denoted $\ulcorner A \urcorner$.
For example, $\ulcorner '0=0' \urcorner = 2430$.

\subsection{Syntactical properties and operations definable in arithmetical language}

Before we go ahead and define other notions using elementary arithmetic, let's introduce
one very useful device: Gödel's $\beta$-function.

\subsubsection{Gödel's $\beta$-function}

$\beta$-function is defined as $\beta(x_1, x_2, x_3) = mod(x_1, 1+(x_3+1)x_2) = 
mod(x_1, x_2x_3+x_2+1)$.

\textbf{$\beta$-function lemma:} for any sequence of natural numbers $(k_0, k_1, ..., k_n)$
there are natural numbers $b$ and $c$ such that for every natural number $0 \leqslant i \leqslant n$,
$\beta(b, c, i) = k_i$.

You will find the proof of this lemma in the appendix, at the end of the document. I suggest you skip it
for now and follow the flow of the main proof. Frankly, I have no clue how Gödel came up with this
function, but the bottom line is we have a device for expressing arbitrary finite sequences of natural
numbers.

We can now use the $\beta$-function to define the $seq$ relation:

\begin{center}
    $\exists := \neg \forall \neg$

    $mod(a, b, c) := \exists_n (a = b \cdot n + c \land c < b)$
    
    $seq(a, b, i, x) := mod(a, (b \cdot i'), x)$
\end{center}

By the $\beta$-function lemma, for every sequence $(k_0, ..., k_n)$ there exist numbers $a, b$ s.t.
$\forall_{0 \leqslant i \leqslant n}(seq(a, b, i, k_i))$.

\subsubsection{Some common formulas}

Let's define some more building blocks of everyday arithmetics:

\begin{center}
    $a \land b := \neg(a \rightarrow \neg b)$

    $a \lor b := \neg a \rightarrow b$
    
    $a \leqslant b := \exists_n (a + n = b \land n \neq 0)$
    
    $a < b := \exists_n (a + n = b \land n \neq 0)$

    $ pow(a, b, c) := \exists_x \exists_y \biggl(
        seq(x, y, 0, 0') \land seq(x, y, b, c) \land
        \forall_k \forall_z \Bigl(
            (k < b \land seq(x, y, k, z))
            \rightarrow
            seq(x, y, k', z \cdot a)
        \Bigl)
    \biggl) $
\end{center}

What happened in that last step? We defined the relation $pow(a, b, c)$ (meaning: $a^b = c$)
as: "there exists a sequence of natural numbers (encoded by $x, y$ as per $\beta$-function
lemma), such that the $0$th element is $1$, and each subsequent element (until $b$-th) is
$a$ times its predecessor, and the $b$-th element equals $c$".

Property of number $x$ being prime:

\begin{center}
    $prime(x) := 0' < x \land \forall_n \forall_m \bigl(
        (0' < n \land n < x \land mod(x, n, m)) \rightarrow \neg (m = 0)
    \bigl)$
\end{center}

Property of number $r$ being $n$-th prime:

\begin{center}
    $prime(r, n) := \exists_x \exists_y \Biggl(
        seq(x, y, 0, 0'') \land seq(x, y, n, r) \land
        \forall_k \forall_s \biggl(
            (k > 0 \land k \leqslant n \land seq(x, y, k, s))
            \rightarrow
            \Bigl(
                prime(s) \land \forall_t \bigl(
                    seq(x, y, k-1, t)
                    \rightarrow
                    \neg \exists_u (t < u \land u < s \land prime(u))
                \bigl)
            \Bigl)
        \biggl)
    \Biggl)$
\end{center}

Or in other words: $r$ is the $n$-th prime, if there exists a sequence (encoded by numbers
$x, y$ as per $\beta$-function lemma) such that the $0$-th element equals $0'' = 2$, $n$-th element
equals $r$, and for each $k \leqslant n$, $k$-th element is a prime bigger than $k-1$-th element,
and there are no primes between $k-1$-th and $k$-th elements.

\subsubsection{$Const(x)$ -- $x$ is (a Gödel number of) a constant}

A constant $n$ is represented in our minimal arithmetics as $0\overbrace{''...'}^{n}$.
Its Gödel number is $p_0^{\#('0')} \cdot p_1^{\#(''')} \cdot ... \cdot p_n^{\#(''')} =
2^1 \cdot 3^2 \cdot ... \cdot p_n^2$. Now we can express it formally:

\begin{center}
    $Const(x) := mod(x, 2, 0) \land \neg mod(x, 4, 0) \land
    \exists_n \forall_m \Bigl(
        \bigl(
            m > 2 \land m \leqslant n \land prime(m)
        \bigl)
        \rightarrow
        \bigl(
            (mod(x, m \cdot m, 0) \land \neg mod(x, m \cdot m \cdot m, 0))
        \bigl)
    \Bigl)$
\end{center}

\subsubsection{$Var(x)$ -- $x$ is (a Gödel number of) a variable}

Recall that symbol number for variable $x_i$ is $11 + i$, so a sequence consisting only of
the variable $x_i$ will be encoded as $p_0^{11 + i} = 2^{11 + i}$.

\begin{center}
    $Var(x) := \exists_n n \geqslant 11 \land pow(2, n, x)$
\end{center}

\subsubsection{Some more properties and relations}

Okay, this is a point where I'll stop presenting exact definitions and start waving my hands
instead. The following relations are conceptually easy to define:

\begin{itemize}
    \item $Length(x, n)$ -- length of the encoded sequence $x$ is $n$. It's enough to check that
        $p_n$ is the biggest prime divisor of $x$.
    \item $Shift(x, n, X)$ -- $X$ is the encoded sequence $x$ shifted to the right by $n$ positions.
    \item $Concat(x, y, z)$ -- $z$ is the encoded concatenation $S_1S_2$, where $x$ is the
        encoded sequence $S_1$ and $y$ is encoded $S_2$. It suffices to check that
        $\exists_{n, Y} \bigl( z = x \cdot Y \land Length(x, n) \land Shift(y, n, Y) \bigl)$.
    \item $CorrectlyParenthesised(x)$ -- $x$ is the encoded correctly parenthesised sequence.
\end{itemize}

\subsubsection{$Term(x)$ -- $x$ is (a Gödel number of) a term}

Note that we cannot use recursion, because all formulas
we come up with must be encodeable with Gödel numbering\footnote{
    A good analogy here are C/C++ macros, which also do not support recursion.
    Consider our helper properties and relations, such as $pow, mod, Term$ etc., "macros"
    to make our minimal arithmetic easier to work with. But ultimately, before evaluating
    a formula, or calculating its Gödel number, we must translate it into primitive
    arithmetic language.
}. If we can't use recursive definition such as "\texttt{Term := Const | Var | Const+Term | ...}",
we'll need to identify what properties of a sequence make it a valid term.

How about:
\begin{center}
    $Term(x) := CorrectlyParenthesised(x)$ and the only allowed transitions between consecutive
    symbols are: \\
    $0 \rightarrow ', +, \cdot, )$ \\
    $' \rightarrow ', +, \cdot, )$ \\
    $x_i \rightarrow ', +, \cdot, )$ \\
    $+ \rightarrow 0, (, x_i$ \\
    $\cdot \rightarrow 0, (, x_i$ \\
    $( \rightarrow 0, (, x_i $ \\
    $) \rightarrow ', +, \cdot, ) $ \\
\end{center}

We'll skip the proof because, again, at this point we're not trying to formally define
logic properties, but just show intuitively that they are defineable if you're patient enough.

\subsubsection{$Form(x)$ -- $x$ is (a Gödel number of) a formula}

Again, a recursive definition would be easy:

\begin{center}
    \texttt{Form := $\neg$ Form | (Form $\rightarrow$ Form) | ($\forall_x$ Form) | (Term = Term)}
\end{center}

But again, we cannot use recursion.

One other trick you may have in mind is, what if we create a sequence of properties
$Form_1(x), Form_2(x), ...$, where $Form_n$ matches formulas of length $n$? The problem here is that
$Form := Form_1 | Form_2 | ...$, it infinitely long, so we can't encode it as a Gödel number.

Instead, let's list properties of a valid formula to see that each of them can be defined without
recursion.
$X$ is a valid formula if (and only if):
\begin{itemize}
    \item $CorrectlyParenthesised(\ulcorner x \urcorner)$
    \item Let $X_i$ denote $i$-th symbol in $X$. For all $i, j$ such that $i < j$ and $X_i = '('$
        and $X_j = ')$:
        \begin{itemize}
            \item if there are no other parentheses in $X_{i + j : j - 1}$, then $X_{i + j : j - 1}$
                is of form $Term = Term$.
            \item if there is equal number of opening and closing parentheses in $X_{i + j : j - 1}$,
                then $X_{i + j : j - 1}$ matches one of the patterns: $\neg (...)$, $(...) \rightarrow (...)$,
                $\forall_x (...)$, where we don't care about content of $(...)$ other than it must be
                $CorrectlyParenthesised$.
        \end{itemize}
\end{itemize}

\subsubsection{$Free(x, y)$ -- $y$ is free variable in $x$}

We can assume, without losing the expressive power of our arithmetic, that we are only concerned with
formulas without variable shadowing.

Then it suffices to check that whenever $y$ occurs in the formula encoded by $x$, it is not directly preceded
by symbol $\forall$.

\subsubsection{$Impl(x, y, z)$ -- implication}

This relation is true for all triplets of formulas $A, B, A \rightarrow B$:
$Impl(\ulcorner A \urcorner, \ulcorner B \urcorner, \ulcorner A \rightarrow B \urcorner)$.

\begin{center}
    $impl(x, y, z) := Form(x) \land Form(y) \land \exists_t \bigl(
        Concat(x, \ulcorner \rightarrow \urcorner, t)
        \land
        Concat(t, y, z)
    \bigl)$
\end{center}

\subsubsection{$subst(x, y, z)$ -- Substitution of free variable}

$subst(x, y, z)$ expresses: \textit{$z$ is (the Gödel number of) the result of substituting the free variable
in (the formula encoded by) $x$ with (the variable encoded by) $y$, assuming that (the formula encoded by)
$x$ has only one free variable}.

For example:

\begin{center}
    $subst(\ulcorner A(x) \urcorner, \underline{n}, \ulcorner A(\underline{n}) \urcorner)$
\end{center}

where $\underline{n}$ is the formal representation of the numeral $n$.

\subsubsection{Formulas for logical derivations}

The $impl$ property we defined earlier can be simply reused to define formula $M$, matching the use
of Modus Ponens:

\begin{center}
    $M(x, y, z) := Form(x) \land Form(z) \land impl(x, z, y)$
\end{center}

Assume that we are working in a formal system $F$ with a fixed set of axioms and rules of derivation.
We can also define:

\begin{itemize}
    \item $LogAx(x)$ -- $x$ is (a Gödel number of) a logical axiom
    \item $Axiom_F(x)$ -- $x$ is (a Gödel number of) a non-logical axiom
    \item $Prf_F(x, y)$ -- $x$ is (a Gödel number of) a derivation (in $F$) of the formula with
        Gödel number $y$
    \item $Prov_F(y) := \exists_x \bigl( Prf_F(x, y) \bigl)$ -- $y$ is (a Gödel number of) a formula
        that is provable in $F$.
\end{itemize}

\section{Arithmetical theories}

\subsection{Q (Robinson arithmetic)}

Axioms:
\begin{center}
    $\neg (0 = x')$ \\
    $x' = y' \rightarrow x = y$ \\
    $\neg (x = 0) \rightarrow \exists y (x = y')$ \\
    $x + 0 = x$ \\
    $x + y' = (x + y)'$ \\
    $x \cdot 0 = 0$ \\
    $x \cdot y' = (x \cdot y) + x$
\end{center}

\subsection{PA (Peano Arithmetic)}

Q + IND axiom:

\begin{center}
    IND: $\phi(0) \land \forall_x \bigl( \phi(x) \rightarrow \phi(x') \bigl) \rightarrow \forall_x \phi(x)$
\end{center}

\section{The Diagonalization Lemma}

\subsection{Warm-up: Cantor's diagonal argument}

Before we immerse ourselves in the Gödelian Diagonalization Lemma,
I'd like to introduce Georg Cantor's so-called diagonal argument.
It's slightly easier to grasp and will at least give you some intuition how
the proof construction works and why it's called diagonal.

We will use Cantor's diagonal argument to prove that there are more real
numbers than natural numbers, or as you may already know, that
natural numbers are countably infinite and real numbers are uncountably
infinite.

Consider the set of all infinite binary sequences, let's call it $T$.
Assume that $T$ is enumerable and pick any enumeration $t_1, t_2, ...$.

For example:

\begin{center}
    $t_1 = (0, 0, 0, 0, 0, 0, 0...)$\\
    $t_2 = (0, 1, 0, 1, 0, 1, 0 ...)$\\
    $t_3 = (1, 0, 1, 0, 1, 0, 1 ...)$\\
    $t_4 = (1, 1, 1, 0, 1, 1, 1 ...)$\\
    $t_5 = (0, 0, 1, 0, 0, 1, 0 ...)$\\
    $t_6 = (0, 1, 0, 0, 1, 0, 0 ...)$\\
    $t_7 = (1, 0, 0, 1, 0, 0, 1 ...)$\\
    ...\\
\end{center}

Now we will show how to construct an infinite sequence which belongs to
$T$ but is not equal to any $t_i$:

\begin{center}
    $s = (1-t_{1,1}, 1-t_{2,2}, ..., 1-t_{i,i}, ...)$
\end{center}

Note that we are picking elements over the diagonal and negating them:

\begin{center}
    $t_1 = (\mathbf{0}, 0, 0, 0, 0, 0, 0...)$\\
    $t_2 = (0, \mathbf{1}, 0, 1, 0, 1, 0 ...)$\\
    $t_3 = (1, 0, \mathbf{1}, 0, 1, 0, 1 ...)$\\
    $t_4 = (1, 1, 1, \mathbf{0}, 1, 1, 1 ...)$\\
    $t_5 = (0, 0, 1, 0, \mathbf{0}, 1, 0 ...)$\\
    $t_6 = (0, 1, 0, 0, 1, \mathbf{0}, 0 ...)$\\
    $t_7 = (1, 0, 0, 1, 0, 0, \mathbf{1} ...)$\\
    ...\\
    $s = (\mathbf{1}, \mathbf{0}, \mathbf{0}, \mathbf{1}, \mathbf{1}, \mathbf{1}, \mathbf{0}, ...)$
\end{center}

Clearly, for any $i$, $s$ and $t_i$ differ on the $i$-th element.
Since $s \in T$ and $s$ is left out of enumeration of $T$, $T$ is not enumerable.

\subsection{The Diagonalization Lemma statement}

\begin{center}
    Let $A(x)$ be arbitrary formula of language of $F$ with only $1$ free variable. Then a sentence $D$ can be
    mechanically constructed such that $F \models \bigl( D \leftrightarrow A(\ulcorner D \urcorner) \bigl)$.
\end{center}

In other words:

\begin{center}
    Let $A$ be arbitrary property formally expressed in $F$. Then it is possible to construct a sentence $D$
    (formally, in $F$), such that $D$ is interpreted in $F$ as "(the Gödelian representation of) $D$ has property $A$".
\end{center}

\subsection{Proof}

Recall the $subst(x, y, z)$ operation that we defined earlier:

\begin{center}
    $subst(x, y, z) \leftrightarrow x = \ulcorner A(x_1) \urcorner \land y = n \land z = \ulcorner A(\underline{n}) \urcorner$
\end{center}

In particular, we can consider formulas of form $subst(x, x, y)$ -- or, if you 
will, $subst(\ulcorner A(x) \urcorner, \ulcorner A(x) \urcorner, y)$.

Given any formula $A(x)$ we can construct formula
$B(x) := \exists_{y} \bigl( A(y) \land subst(x, x, y) \bigr)$
with one free variable $x$. Like every formula in $F$, $B(x)$ has a Gödel number.
Let's denote it as $k := \ulcorner B(x) \urcorner$.

Now, substitute $x$ in $B$ with $\underline{k}$. Name the resulting sentence $D$:

\begin{center}
    $D := B(\underline{k}) := \exists_y \bigl(
        A(y) \land subst(\underline{k}, \underline{k}, y)
    \bigr)$
\end{center}

If you're not confused now, you should be. Let me unwrap this definition above
to show deeply we've gone with encoding formulas into Gödel numbers:

\begin{center}
    $B(\underline{k}) \leftrightarrow \\
    B(\ulcorner B(x) \urcorner) \leftrightarrow \\
    B(\ulcorner \exists_y \bigl(
        A(y) \land subst(x, x, y)
    \bigr) \urcorner) \leftrightarrow \\
    \exists_y \Bigl(
        A(y) \land subst(
            \ulcorner \exists_z \bigl(subst(x, x, z) \bigr) \urcorner,
            \ulcorner \exists_z \bigl(subst(x, x, z) \bigr) \urcorner,
            y)
    \Bigr)
    $
\end{center}

\section{Appendix}

\subsection{$\beta$-function lemma proof}

To prove the $\beta$-function lemma, we will take advantage of Chinese Remainder Theorem (CRT):

\begin{center}
    \textit{
        Suppose $x_0, ..., x_n$ are pairwise relatively prime. Let $y_0, ..., y_n$ be any natural numbers.
        Then there is a number $z$ such that:
    }

    $z \equiv y_0\ \text{mod}\ x_0$\\
    ...\\
    $z \equiv y_n\ \text{mod}\ x_n$
\end{center}

Now the trick is to find such sequence $x_0, ..., x_n$ that can be used in our limited arithmetic
without remembering the whole sequence as variables and that will guarantee existence of number $z$
which solves the CRT system of congruences, thus encoding our original sequence $y_0, ..., y_n$.
Our sequence $x_0, ..., x_n$ should also satisfy $x_i > y_i$, so that given the number $z$,
sequence $y$ will be uniquely decodable.

Given $y_0, ..., y_n$, let:

\begin{center}
    $j = \text{max}(n, y_0, ..., y_n) + 1$
\end{center}

and:

\begin{center}
    $x_0 = 1 + j!$\\
    $x_1 = 1 + 2 \cdot j!$\\
    ...\\
    $x_n = 1 + (n + 1) \cdot j!$
\end{center}

This sequence satisfies the conditions of CRT, because:

\begin{itemize}
    \item $x_i > y_i$. More precisely: $x_i > j! > j \geqslant y_i$.
    \item $x_i$ are pairwise relatively prime. If they weren't there would be such prime number $p$
        and such $x_i, x_j, x_i \leq x_j$ that $p | x_i$ and $p | x_j$. If $p | x_i$ and $p | x_j$
        then $p | x_i - x_j$, so $p | m \cdot j!$ for some $m \leqslant n$. $p$ cannot divide $j!$
        because it divides $x_i = 1 + (i+1) \cdot j!$, so it must divide $m$. But $m \leqslant j$,
        so it must also divide $j!$. We have a contradiction, therefore such $p$ cannot exist.
\end{itemize}

Thus, for any sequence of natural numbers $y_0, ..., y_n$, there exists number $z$
such that $\forall_{0 \leqslant i \leqslant n}(\beta(z, j!, i) = mod(z, 1 + (i + 1) \cdot j!) = y_i)$,
where $j$ is constructed as above.

\subsection{CRT proof}



% TODO: bilbiography
% https://plato.stanford.edu/entries/goedel-incompleteness/#FirIncThe
% https://builds.openlogicproject.org/content/incompleteness/representability-in-q/beta-function.pdf

\end{document}